	\documentclass[12pt,letter]{article}
\usepackage{amsmath}
\usepackage{amssymb}	% packages that allow mathematical formatting
\usepackage{graphicx}	% package that allows you to include graphics
\usepackage{setspace}	% package that allows you to change spacing
\usepackage{fullpage}	% package that specifies normal margins
\usepackage{microtype}
\usepackage{amsthm}
\newcommand{\argmin}{\operatornamewithlimits{argmin}}
\renewcommand\qedsymbol{$\blacksquare$}
\usepackage{listings}
\usepackage{color}

\definecolor{codegreen}{rgb}{0,0.6,0}
\definecolor{codegray}{rgb}{0.5,0.5,0.5}
\definecolor{codepurple}{rgb}{0.58,0,0.82}
\definecolor{backcolour}{rgb}{0.95,0.95,0.95}

\lstdefinestyle{mystyle}{
	backgroundcolor=\color{backcolour},   
	commentstyle=\color{codegreen},
	keywordstyle=\color{magenta},
	numberstyle=\tiny\color{codegray},
	stringstyle=\color{codepurple},
	basicstyle=\footnotesize,
	breakatwhitespace=false,         
	breaklines=true,                 
	captionpos=b,                    
	keepspaces=true,                 
	numbers=left,                    
	numbersep=5pt,                  
	showspaces=false,                
	showstringspaces=false,
	showtabs=false,                  
	tabsize=2
}

\lstset{style=mystyle}

\usepackage[left=2.5cm, right=2.5cm, top=2cm, bottom = 3cm]{geometry}


	

\begin{document}
\title{FNCE Problem Set 1}
\author{Patrick Shultz and Felix Nockher}
\date{\today}
\maketitle 
\section*{Problem 1}
\paragraph{a)}
Consumer maximizes
\begin{equation*}
	U = E_0\left[ \sum_{t = 0}^{\infty} \beta^t( u(a+ \theta_t)c_t - bc_t^2)\right] 
\end{equation*}
subject to $c_t = Rk_t + y_t - k_{t+1}$. The objective function and budet constraint imply the Lagrangian is
\begin{equation}
	\mathcal{L} = u(c) +\beta E\left[V(k', y', R') \right] + \lambda \left[Rk + y - c'-k' \right]  
\end{equation}
First order conditions and the envelope condition (3rd line) imply
\begin{equation}
	\begin{split}
	 u_c(c) - \lambda &=0\\
	\beta E_{y, R}\left[ V_k(k', y', R')\right]-\lambda &= 0\\
	 V_k(k, y, R) &= R\lambda\\  
	\end{split}
\end{equation}
The FOCs with respect to $c$ and the Envelope condition with respect to $k$ imply the Euler equation 
\begin{equation}
	u_c(c) = \beta E_{y, R} V_k(k', y', R)
\end{equation}
Iterating forward one time period on the Envelope condition gives
\begin{equation}
\begin{split}
u_c(c) &= \beta E_{y, R} R u_c(c')\\
&= E_{y} u_c(c')\\
\end{split}
\end{equation}
where $u_c(c) = (a+\theta) + 2bc$, so our Euler equation is 
\begin{equation}
a + \theta_t - 2bc_t = E_t\left[a + \theta_{t+1} -2bc_{t+1} \right]  
\end{equation}

\paragraph{b)} Assuming $\theta_t = \phi \theta_{t-1} + \epsilon_t$ and for convergence assuming also $|\phi| <1$, we can rewrite the Euler equation as 
\begin{equation}
\begin{split}
	c_t & = E_t c_{t+1}+\bigg(\frac{(1-\phi)\theta_t}{2b}\bigg)\\
	&=E_t\left[  c_{t+2} + \bigg(\frac{(1-\phi)\theta_{t+1}}{2b}\bigg)\right] +\bigg(\frac{(1-\phi)\theta_t}{2b}\bigg)\\
	&= E_t\left[  c_{t+s}\right] +\bigg(\frac{(1-\phi)}{2b}\bigg)\sum_{s = 1}^{\infty}\phi^{s-1}\theta_t\\
	&= E_t\left[  c_{t+s}\right] +\bigg(\frac{(1-\phi)}{2b}\bigg)\frac{\theta_t}{1-\phi}\\
	&= E_t\left[c_{t+s}\right] + \frac{\theta_t}{2b}\\
\end{split}
\end{equation}
which implies $E_0\left[c_{t}\right] = c_0 -\frac{\theta_0}{2b}$. 
We now consider the lifetime budget constraint to characterize the process for optimal consumption. 	
\begin{equation}
\begin{split}
Rk_0 + \sum_{j = 0}^{\infty}R^{-t}E_0\left[ y_t\right] &= \sum_{t=0}^{\infty}R^{-t}E_0\left[c_t\right]\\
\implies Rk_0 + \sum_{t=0}^{\infty}R^{-t}E_0\left[y_t\right]  &= \frac{R}{R-1}\bigg(c_0 -\frac{\theta_0}{2b}\bigg)\\
\implies \frac{R-1}{R}\bigg(Rk_0 + \sum_{t = 0}^{\infty}R^{-t}E_0\left[y_t\right]\bigg) +\frac{\theta_0}{2b} &= c_0\\
\end{split}
\end{equation}
where the second line holds from the Euler equation. If we assume the process of $y$ satisfies the Markov property, we have characterized consumption as a function of current state variables. 
\paragraph{c)} We now assume $\theta$ follows a random walk, $\theta_{t+1} = \theta_t + \epsilon_{t+1}$. We can rewrite the Euler equation as
\begin{equation}
\begin{split}
	a + \theta_t - 2bc_t &= E_t\left[a + \theta_{t+1} - 2bc_{t+1}\right]\\
	&=  E_t\left[a + \theta_{t}+ \epsilon_{t+1} - 2bc_{t+1}\right]\\
	\implies c_t &=E_t\left[c_{t+1}\right]
\end{split}
\end{equation}
Thus, when $\theta$ follows a random walk, we see that we are in the case that $c$ is a random walk. Plugging this condition into the budget constraint, we get
\begin{equation}
	\begin{split}
	Rk_0 + \sum_{t = 0}^{\infty}R^{-t}E_0\left[y_t\right] = \sum_{t = 0}^{\infty}R^{-t}E_0\left[c_t\right]\\
	Rk_0 + \sum_{t = 0}^{\infty}R^{-t}E_0\left[y_t\right] = \frac{R}{R-1}c_0\\
	c_0 = \frac{R-1}{R}\bigg(Rk_0 + \sum_{t = 0}^{\infty}R^{-t}E_0\left[y_t\right]\bigg)
	\end{split}
\end{equation}
Once again, we assume $y$ satisfies the Markov property.
\paragraph{d)} Assume $\beta R < 1$  and $\theta$ is a random walk. The Euler equation gives
\begin{equation}
\begin{split}
	a + \theta_t + 2bc_t = \beta R E_t\left[a + \theta_{t+1} + 2b c_{t+1}\right]\\
	\implies -2bc_t = a(\beta R -1) + \theta_t(\beta R -1) - 2bE_t\left[c_{t+1}\right]\\
	\implies c_t = \bigg(\frac{1-\beta R}{2b}\bigg)(a + \theta_{t}) + E_t\left[c_{t+1}\right]
	\end{split}
\end{equation}
Iterating forward on $ c_t = \bigg(\frac{1-\beta R}{2b}\bigg) + E_t\left[c_{t+1}\right]$ yields 
\begin{equation}
	c_t = E_t\left[c_{t+s}\right] + s\bigg(\frac{1-\beta R}{2b}\bigg)(a + \theta_{t})
\end{equation}
Intuitively, this implies that when $\beta R < 1$ the rate of return on savings is too low, so an agent will consume rather than invest. In our expression for $c_t$, we can see that consumption in later periods is lower than consumption in earlier periods. Plugging into the lifetime budget constraint, we get
\begin{equation*}
\begin{split}
	Rk_0 + \sum_{t = 0}^{\infty}R^{-t}E_0\left[y_t\right] &= \sum_{t = 0}^{\infty}R^{-t}E_0\left[c_t\right]\\
	Rk_0  + \sum_{t = 0}^{\infty}R^{-t}E_0\left[y_t\right] &= \sum_{t = 0}^{\infty}R^{-t}\bigg(c_0 - t \bigg(\frac{1-\beta R}{2b}\bigg)(a + \theta_0)\bigg)\\
	Rk_0  + \sum_{t = 0}^{\infty}R^{-t}E_0\left[y_t\right] &= \frac{R}{R-1}c_0 - \sum_{t = 0}^{\infty}R^{-t}\bigg(t \bigg(\frac{1-\beta R}{2b}\bigg)(a + \theta_0)\bigg)\\
\end{split}
\end{equation*}
Thus, we get that our consumption process is given by
\begin{equation}
\begin{split}
	c_0 &= \frac{R-1}{R} \bigg[Rk_0  + \sum_{t = 0}^{\infty}R^{-t}E_0\left[y_t\right] + \sum_{t = 0}^{\infty}R^{-t}\bigg(t \bigg(\frac{1-\beta R}{2b}\bigg)(a + \theta_0)\bigg)\bigg]\\&= \frac{R-1}{R} \bigg[Rk_0  + \sum_{t = 0}^{\infty}R^{-t}E_0\left[y_t\right] + \frac{R}{(R-1)^2}\bigg( \bigg(\frac{1-\beta R}{2b}\bigg)(a + \theta_0)\bigg)\bigg]
\end{split}
\end{equation}
\section*{Problem 2}
\paragraph{a)}
Recursive formulation via Bellman equation:
\begin{equation} \label{Problem2}
V(h_t;A_t) = \underset{c_t,h_{t+1}}{\max}\{log(c_t) + \beta E_t\left[ V((h_{t+1};A_{t+1}))\right] \} \text{ s.t. } c_t = A_th_t-h_{t+1}
\end{equation}

\paragraph{b)}
We ought to guess/ verify the following form of the value function: $V(h;A) = a/ log(h) + v(A)$. We start by assuming it holds for $t+1$ and then show it also holds for $t$:
\begin{align*}
V(h_t;A_t) 		&= \underset{c_t,h_{t+1}}{\max}\{log(c_t) + \beta E_t\left[ a\ log(h_{t+1}) + v(A_{t+1}))\right] \}\\
&= \underset{c_t,h_{t+1}}{\max}\{log(c_t) - log(h_t) + log(h_t) + \beta E_t\left[ a\ (log(h_{t+1}) - log(h_t) + log(h_t)) + v(A_{t+1}))\right] \}\\
&= (1+\beta a)log(h_t) + \underset{\hat{c}_t,\hat{h}_{t+1}}{\max}\{log(\hat{c}_t) + \beta E_t\left[ a\ log(\hat{h}_{t+1}) + v(A_{t+1}))\right] \} &(\star)\\
&= \underbrace{(1+\beta a)}_{\equiv a} log(h_t) + \underbrace{ \beta E_t[v(A_{t+1})] + \underbrace{\underset{\hat{c}_t,\hat{h}_{t+1}}{\max}\{log(\hat{c}_t) + \beta E_t\left[ a\ log(\hat{h}_{t+1}))\right] \}}_{\equiv g(A_{t})}}_{\equiv v(A_t)}
\end{align*}
$(\star)$ introducing the following changes of variables: $\hat{c}_{t} = \frac{c_{t}}{h_{t}}$ and $\hat{h}_{t+1} = \frac{h_{t+1}}{c_{t}}$\\
We clearly see that $a$ is a constant: $a = \frac{1}{1-\beta}$. Now, we conjecture that $v(A_t) = E_t \left[\sum_{s=0}^{\infty}\beta^s g(A_{t+s})\right]$ where, as indicated above, $g(A_{t})=\underset{\hat{c}_t,\hat{h}_{t+1}}{\max}\{log(\hat{c}_t) + \beta E_t\left[ a\ log(\hat{h}_{t+1}))\right] \}$.
We again verify this by assuming that it holds for $t+1$ and show it holds for $t$:
\begin{align*}
\beta E_t(v(A_{t+1}))+g(A_t) 	&= \beta E_t\left[ E_{t+1}\left[ \sum_{s=0}^{\infty}\beta^s g(A_{t+1+s}) \right] \right] +g(A_t)\\
&= \beta E_t\left[\beta^{-1}g(A_t) + \sum_{s=0}^{\infty}\beta^s g(_{t+1+s}) \right]\\
&= E_t \left[g(A_t) + \sum_{s=0}^{\infty} \beta^{s+1} g(A_{t+1+s}) \right]\\
&= E_t \beta^s \sum_{s=0}^{\infty}g(A_{t+s})\\
&= v(A_t) \ _\blacksquare
\end{align*}
Note that this representation of $v(A_t)$ is only true if $E_t \beta^s \sum_{s=0}^{\infty}g(A_{t+s}) < \infty$ which depends on the distribution of $A$.

\paragraph{c)}
For the optimal consumption and human capital investment choices we deploy the FOC w.r.t. $c_t$, see eq. \ref{FOC},  and the envelope condition, see eq. \ref{Env}, where $\lambda$ is the Lagrange multiplier from rewriting eq. \ref{Problem2} as a constrained maximization:
\begin{equation}\label{FOC}
\frac{\partial V}{\partial c_t}: 0 = u_c(c_t) - \lambda
\end{equation}
\begin{equation}\label{Env}
\frac{\partial V}{\partial h_t} = \lambda A_t\\
\end{equation}
Combining eq. \ref{FOC} and \ref{Env} and using our newly confirmed form of the value function we then get:
\begin{align*}
u_c(c_t)		&= V_h(h_t;A_t)\frac{1}{A_t}\\
&= \frac{\partial}{\partial h_t}(a\ log(h_t) + v(A_t))\frac{1}{A_t}\\
\frac{1}{c_t}	&= a\frac{1}{h_t}\frac{1}{A_t}
\end{align*}
Which then with the fact that $a = \frac{1}{1 - \beta}$ gives the optimal consumption as a function of current state variables and known parameters:
\begin{equation}
c_t^\star = A_t h_t (1-\beta)
\end{equation}
Using $c_t^\star$ in the budget constraint then yields optimal next period human capital investment as a function of the current state variables and known parameters:
\begin{equation}
h_{t+1}^\star = A_t h_t \beta 
\end{equation}

\paragraph{d)}
Knowing that $log A \sim^{iid} N(\mu_A,\sigma^2)$ we can use our expression for next period's optimal human capital investment to find the expected (gross) growth rate of human capital investments:
\begin{align*}
h_{t+1} 								&= \beta A_t h_t\\
\frac{h_{t+1}}{h_t} 					&= \beta A_t\\
log\left[\frac{h_{t+1}}{h_t}\right] 	&= log\beta +log A_t\\
E_t log\left[\frac{h_{t+1}}{h_t}\right] 	&= log\beta + E_t log A_t\\
E_t log\left[\frac{h_{t+1}}{h_t}\right] 	&= log\beta + \mu_A
\end{align*}
For the consumption growth process we make use of the Euler equation $u_c(c_t) = \beta E_t [A_{t+1}u_c(c_{t+1})]$:

\begin{align*}
\frac{1}{c_t} 	&= \beta E_t\left[ A_{t+1} \frac{1}{c_{t+1}} \right]\\
1				&= \beta E_t\left[ A_{t+1} \frac{c_t}{c_{t+1}} \right]\\
&= \beta E_t\left[ A_{t+1}\right] E_t\left[ \frac{c_t}{c_{t+1}} \right] &\text{due to iid'nes of A}\\
E_t\left[ \frac{c_{t+1}}{c_{t}} \right] &= \beta E_t\left[ A_{t+1} \right]\\
log E_t\left[ \frac{c_{t+1}}{c_{t}} \right] &= log \beta + \mu_A + \frac{1}{2}\sigma^2
\end{align*}

\end{document}


